\font\modernlarge=cmss17
\def\blank{\vskip .2 in\filbreak}

{\modernlarge\centerline{Welcome to the world of mdraw V4.0}}
\medskip
This document is aimed toward the VT330 owner, but a VT240 owner may still learn
something. Most commands are available from the keyboard as well as the menu, so
if you don't have a mouse, that is okay. 

When the instructions say to select the
menu item for a command (e.g. `4' to draw a rectangle) that menu item is a 
character, so you may alternately press that same character on the keyboard 
for the same effect.

See the `start' file for information on setting up the program and logicals.

\blank
{\modernlarge \item{0.}{Introduction}}

The world of mdraw consists of three things. These are the workspace, objects 
and actions. You can manipulate the workspace with actions, such as 
changing global values like linestyle, gridsize and textsize or deleting all 
objects in the workspace. You can manipulate
objects with actions, such as creating objects, moving, cutting, joining and changing
linestyles. The actions act upon objects or the workspace. If an action is meant
for objects, you must have an object selected. If it is meant to alter the
workspace it usually does not matter if an object is selected. 
There are some actions
which alter the selected object, and if there is no selected object, it will
alter the workspace.

\blank
{\modernlarge \item{1.}{Creating Objects}}

Mdraw can draw several objects. Lines, circles, boxes, splines, and text.
To draw an object, you select the menu item for that object. The menu items
for each object are listed below. The menu 
item you have selected will light up. Move the mouse to
the destination and press a button. If the object needs more info, such as
the opposite corner for a box, the menu item will stay lit, and the mouse
cursor will be a rubber band box or line, depending on the object. Keep moving
the mouse and entering new points of information until you are done with the
object. To enter a new point, press any key or mouse button. If you are drawing
a spline, you may finish entering points by pressing $<$space$>$ or the left mouse
button. The menu item will return to normal, and the completed object will be
placed on the workspace. This object will be automatically selected for any 
further action by a dotted box being drawn around it. (if the object is a box
the selection box will be drawn exactly on top of the object box, obscuring it
until deselection.)

\blank
{\modernlarge \item{2.}{Selecting objects }}

To select an object from the workspace, move the mouse
cursor so that it is on top of the object you want and press the middle button.
To deselect an object, move to a place where there is no object and press the
middle button.
When an object is selected, a dotted box will be drawn around the object.
Several objects may occupy the same place on the workspace. Imagine these
objects to be stacked above or below each other like cards in a deck. 
To select an object below
another object, press the button (or key) several times, until the right object
is selected. If there are no more objects below, then no object is selected,
effectively deselecting the last object. You may use the keyboard and press 
the `$<$find$>$' key or the
`$<$select$>$' key or the `f' key. These keys work slightly different however. See
below for more info. You may select objects above or below in this stack of
objects with the `[' key and the `]' key. `[' moves up the stack, `]' moves
down. 

The `$<$find$>$' key finds the top object in the stack, the `$<$select$>$' key
selects the next object down in the stack.

\blank
{\modernlarge \item{3.}{Actions}}

Once an object is selected from the workspace, mdraw allows a several
actions which change the object in relation to the workspace. 
They are: move, join, unjoin, cut, and paste. Mdraw allows some action which alter the objects
characteristics, such as its linestyle, and its associated file. You may also
show characteristics of an object by pressing the `W' key when one is selected.
All these actions except join must have a selected object to work on. So select an object
before you try them out. 

To move an object, select it with the middle button, move the
mouse to the destination, press the left button (or press the `m' key).

To join objects, a rectangle will be drawn which surrounds the objects to be
joined. First move the cursor to one of the intended rectangle, press `j', move
to the opposite corner of the joining rectangle and press a key. Objects `under'
this rectangle will bejoined and a single object will be formed, this object
will be selected and the bounding box drawn around it. The way mdraw determines
`under' is if a corner of the joining rectangle is inside the objects bounding
box, or the objects corner is inside the joining rectangle. This new joined
object cannot have an associated file, but instead uses the associated file of
one its sub-objects (as long as the sub-object is a primitive object and not a
joined object). If it doesn't look right you can unjoin it and try again. 

To unjoin and object, select an object with the middle button, press `J'. The
component objects will be unjoined and placed in the workspace, and no object
will be selected.

To cut an object, select it with the middle button, select `c' from 
the menu, or press `c' on the keyboard, or press the `$<$remove$>$' key.

To paste an object, it must have been cut first. Select `p' from the
menu, move to the destination and press a button. Or you may move to the
destination then press the `p' key or the `$<$insert here$>$' key.

To change linestyle, select the object, press the `\_' (underline) key.
A prompt will tell you the present line style and you may type in the new 
style number. The object will be redrawn immediately in the new style.
Note: if the object is a box, the dotted selection box will cover up the new 
linestyle, and you should deselect the object to see the change.

To change the associated file, select an object, and press the `A' key.
A prompt will tell you the present file, and you may enter the new name,
including the file extension. It is important to include the file extension.

\blank
{\modernlarge \item{4.}{Workspace}}

The workspace is the place where objects are placed when created or
pasted and where the menu is displayed. When an object is placed there, it 
may be constrained to some workspace
attributes. These are: Grid, linestyle, and fill. 

Grid constrains all cursor input to the intersections of a
grid. This may be toggled on and off by pressing the `g' key.
This grid size may be changed. Press `G' key and a prompt will tell you
the current size and you may enter the new size.

Linestyle determines the linestyle for all created (not pasted) objects,
and may also be changed. Press the `\_' key when NO objects are selected, and a
prompt will show the current style and you may enter the new style number. The
menu will be immediately redrawn in the new style for visual confirmation.
	
Fill determines if boxes and circles are drawn filled in. This may be
toggled on and off by pressing the `F' key.

Information about the status of the workspace may be shown by pressing
the `W' key.
	
The menu may be toggled by pressing the `F20' key. (upper right of
keyboard).

\blank
{\modernlarge \item{5.}{Objects }}
\item{Circle:}{Press `0', move mouse to center, press button, move mouse to
edge, press button. You may create a filled circle by turning on fill (press
`F') before creating circle.}
\item{Line:}{Press `1', move mouse to start point, press button, move mouse to
end point, press button. }
\item{Box:}{Press `4', move mouse, press button, move mouse, press button. You
may create a filled box by turning fill on (press `F') before creating box. }
\item{Spline:}{Press `s', move mouse to first point, press button, move mouse, 
to finish spline press left mouse button (or spacebar), any other button to 
enter next point. You may have arrowheads on splines by turning on arrows (press
`k'). }
\item{Text:}{Press `t', move mouse to destination, press button, type in text,
hit return. You may change text size by pressing `T' and answering prompt. }

\blank
{\modernlarge \item{6.}{Miscellaneous}}
\item{X:}{Delete all objects from the workspace, but not cut buffer.
Automatically sets the zoom file to previously loaded or saved filename.}
\item{I:}{Toggle mouse cursor shape}
\item{C:}{Use current mouse position as new center of screen and move all
objects to reflect this}
\item{u:}{Delete top item on the workspace}
\item{r:}{Redraw screen}
\item{R:}{Redraw grid (if it is on)}
\item{$<$help$>$:}{Draw help screen}
\item{z:}{Zoom up to parent file (Note this will lose any changes not saved)}
\item{x:}{eXplode an object to get to the associated mdraw file}
\item{A:}{Associate object with a file. (Note: the object must be a primitive 
object, i.e. not a joined object.)}
\item{Z:}{Set the zoom file for the present file. You should then save the file
to keep this change.}
\item{Q:}{Quit}

\bye
                                                   
