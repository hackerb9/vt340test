\font\modernlarge=cmss17
\def\blank{\vskip .2 in\filbreak}

$$\vbox{\input mdraw.tex }$$

{\modernlarge\centerline{Startup Information}}

\blank
\item{}{Programs needed}
\item{mdraw.exe}{- Object oriented drawing program}
\item{vregis.exe}{- convert mdraw file to `.pic' screen dump and/or \TeX\ file}
\item{sixel.com}{- Runs one of the following screen dump programs:}
\itemitem{sixland.exe}{- Landscape screen dump}
\itemitem{sixel.exe}{- Portrait screen dump}

\blank
\item{}{Files needed}
\item{mdraw.mdr}{- Title screen}

\blank
\item{}{Logicals and symbols needed}
\item{mdraw}{- `assign $<$disk-directory-path$>$ mdraw'}
\item{mdraw}{- `mdraw :== \$mdraw:mdraw.exe'}
\item{sixel}{- `sixel :== @mdraw:sixel.com'}
\item{vregis}{- `vregis :== \$mdraw:vregis.exe'}

\blank

\blank
Before drawing, make sure the terminal can draw ReGIS graphics, and make sure
that it is in VT300-8Bit mode from the setup menu. To start drawing, just type
`mdraw' at the prompt, and you will see a title screen which displays the
version number. Press a key (but not return) or mouse button and the screen will
clear and crosshairs will appear. You may now begin entering commands, either
from the keyboard or through the mouse and menu. To turn the menu on, press the 
F20 key

Once you have an mdraw file (with the extension `.mdr') that you want to print
out in one form or another, you can call the `vregis' command.
The mdraw to sixel conversion program `vregis' has command line parameters:

`vregis [-x] [-t] [-l] [-o] [-v] file[.ext] (...file[.ext]...) '. 

The `-v' will just view the picture without doing any screen dump, so there
will be no raw `.pic' file.

The `-x' means to output a \TeX\ file, this file may be referenced from another
\TeX\ file with the $\backslash$input command.

The `-o' means to output a \TeX\ file and also includes commands to draw a
box around the image. This file may be included in any other \TeX\ file just
like the -x option. 

The `-t' is like `-x' but has \TeX\ handle the smaller fonts, (which are:
cmr5, cmtt8, cmtt12 scaled magstep1 and cmtt12 scaled magstep4 )
but it doesn't handle math characters, so you'll have to go in and put dollar signs
around all the funny symbols.

The `-l' means to output a landscape `.pic' file, with no \TeX\ file. 

The command defaults to look for a `.mdr'
file, and if you give it a different extension the rest of the command line will be
assumed to be that type file, until another file extension is found. 

Remember! you must have the VT330 in `VT300-8bit' in the General Set-up of
the set-up menu for the mouse to work properly, if you don't you will get a 
message and the program might crash.

\bye
